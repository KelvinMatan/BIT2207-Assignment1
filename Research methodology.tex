 \documentclass[11pt]{article}

\usepackage[top=1in, bottom=1in, left=1in, right=1in]{geometry}

\begin{document}

\textbf{MATANDA KELVIN}


\textbf{REGISTRATION NUMBER: 15/U/7444/EVE}


\textbf{STUNDENT NUMBER: 215004160}


\textbf{RESEARCH REPORT}\\



\begin{huge}\textbf{LIVING IN A SLUM}\end{huge}\\

\section{INTRODUCTION}

This document gives a brief overview of what to expect when you are to inhabit in a slum commonly known as the “ghetto” present day. 
Almost everything that is said about living in a slum makes people fear and never consider staying there even if they are financially lacking. 
Previous research shows that there is no good that comes out of such places but there is.

\section{OBJECTIVES}
The main aim of this research is to find out whether there are any advantages to staying in a poor and densely populated neighbourhood commonly known as the ghettos or slums.
Another objective is to state clearly the disadvantages of staying in one.

\section{RESEARCH METHODOLOGY}
The easiest method and the one that I used a lot was observation.
Asking questions is another method I used.
The last one was facts from already done research by some other people.

\section{RESULTS/FINDINGS}
As regards to the research methods stated, a lot was found out about the living in a slum. 
The advantages are not as many but are substantial and some of them are as stated below;

\begin{enumerate}

\item An extremely low cost of living: people survive on as little as two thousand Ugandan shillings a day
. In an elite neighbourhood for example maybe Kololo, that little money cannot buy you a tomato to supplement a meal. In areas like Katanga,
 a person wakes up to a breakfast of about five hundred shillings, lunch of about two thousand so is the supper.

\item Easy socialising and making friends: One can be known by the entire village since people live in masses in
 a small geographical area. A 50 by 50 plot of land can have around 10 to 15 occupants. Tell me how you fail to make friends.

\item Slums are also areas of happiness. People are high as early as 8:00 am in the morning.
 There are just too many happening places in the form of bars, meat roasting joints, lodges, betting joints, movie halls and so much more. 

\end{enumerate}•
However just as there are advantages, the disadvantages are many and below I will sate some of them;

\begin{enumerate}

\item High crime rates: This is the biggest issue with such areas. This rises from the fact that most people are unemployed
 but still the need to survive. So the security in these areas is very low.

\item The congestion of people in a small place also comes with its issues the biggest one being poor hygiene. That’s the reason most epidemics
 diseases break from such areas.

\item Poor facilities due to under development: There are little or no facilities in such places such as Hospitals, Schools, good roads etc.
 Most people are contented with what they have and so do not complain hence the reason government has also not done a lot to help them.

\end{enumerate}•

\section{CONCLUSION}
It was found out that much as the slums are not the best places to be, they can be good as well not disregarding the
 fact that they are also not the best places to be or stay.

\section{REFERENCES}
All this information came from my head. So these are the views not of the general public but me as a person. 



\end{document}
